%-----------------------------------------------------------------------
%-----------------------------------------------------------------------
%
%   This is a collection of TEX-macros taken from the former  PHYZZL
%   adjusted to work under TEX-version 1.0 .
%
%   Last changed on  January 27th, 1986    by   Ewald Mueller
%
%   Last changed on  February 25th, 1992    by   S.Blinnikov
%
%-----------------------------------------------------------------------
%-----------------------------------------------------------------------
%
%
% Load fonts with baselineskip 18pt   (default is 12pt)
%
%
%-----------------------------------------------------------------------
%   Additional fonts are loaded.  Notice that the fonts
%      \sevenrm, \seveni, \sevensy  and  \sevenbf
%   are preloaded in PLAINORG TEX
%
%   font ambc12 at 14.4treupt changed to ambx10 at 14.4truept on
%   October 29th, 1985   E.Mueller
%-----------------------------------------------------------------------
%
\font\twelverm = cmr10  at 12truept
\font\ninerm   = cmr9
%
\font\twelvei  = cmmi10 at 12truept
\font\ninei    = cmmi9
%
\font\twelvesy = cmsy10 at 12truept
\font\ninesy   = cmsy9
%
\font\twelvebf = cmbx10 at 12truept
\font\ninebf   = cmbx9
%
\font\twelvex  = cmex10 at 12truept
\font\twelveit = cmti10 at 12truept
\font\twelvesl = cmsl10 at 12truept
\font\twelvett = cmtt10 at 12truept
%
%   A few more fonts are needed for special purposes
%
\font\bigbc = cmbx10  at 14.4truept    % Institut-logo for letters
\font\caps  = cmcsc10 at 12.0truept    % names of authors and sections
%
%-----------------------------------------------------------------------
%   Size-switching macro
%-----------------------------------------------------------------------
%
\def\twelvepoint{   \def\rm{ \fam0 \twelverm }
  \textfont0=\twelverm \scriptfont0=\ninerm  \scriptscriptfont0=\sevenrm
  \textfont1=\twelvei  \scriptfont1=\ninei   \scriptscriptfont1=\seveni
  \textfont2=\twelvesy \scriptfont2=\ninesy  \scriptscriptfont2=\sevensy
  \textfont3=\twelvex  \scriptfont3=\twelvex \scriptscriptfont3=\twelvex
  \textfont\itfam=\twelveit         \def\it{ \fam\itfam \twelveit }
  \textfont\slfam=\twelvesl         \def\sl{ \fam\slfam \twelvesl }
  \textfont\ttfam=\twelvett         \def\tt{ \fam\ttfam \twelvett }
  \textfont\bffam=\twelvebf  \scriptfont\bffam=\ninebf
  \scriptscriptfont\bffam=\sevenbf  \def\bf{ \fam\bffam \twelvebf }
  \normalbaselineskip=18pt
  \normalbaselines \rm   }
%
%\font\bmit=cmbxti10
\font\tenmbit=cmbxti10 at 12truept

\newfam\mbifam
\textfont\mbifam=\tenmbit

\def\bmit{\fam\mbifam\tenmbit}

\def\vector#1{\bmit #1}
%
\def\getsto{\mathrel{\mathchoice {\vcenter{\offinterlineskip
\halign{\hfil$\displaystyle##$\hfil\cr\gets\cr\to\cr}}}
{\vcenter{\offinterlineskip\halign{\hfil$\textstyle##$\hfil\cr
\gets\cr\to\cr}}}
{\vcenter{\offinterlineskip\halign{\hfil$\scriptstyle##$\hfil\cr
\gets\cr\to\cr}}}
{\vcenter{\offinterlineskip\halign{\hfil$\scriptscriptstyle##$\hfil\cr
\gets\cr\to\cr}}}}}
%
\def\la{\mathrel{\mathchoice {\vcenter{\offinterlineskip\halign{\hfil
$\displaystyle##$\hfil\cr<\cr\noalign{\vskip1.5pt}\sim\cr}}}
{\vcenter{\offinterlineskip\halign{\hfil$\textstyle##$\hfil\cr<\cr
\noalign{\vskip1.0pt}\sim\cr}}}
{\vcenter{\offinterlineskip\halign{\hfil$\scriptstyle##$\hfil\cr<\cr
\noalign{\vskip0.5pt}\sim\cr}}}
{\vcenter{\offinterlineskip\halign{\hfil$\scriptscriptstyle##$\hfil
\cr<\cr\noalign{\vskip0.5pt}\sim\cr}}}}}
%
\def\ga{\mathrel{\mathchoice {\vcenter{\offinterlineskip\halign{\hfil
$\displaystyle##$\hfil\cr>\cr\noalign{\vskip1.5pt}\sim\cr}}}
{\vcenter{\offinterlineskip\halign{\hfil$\textstyle##$\hfil\cr>\cr
\noalign{\vskip1.0pt}\sim\cr}}}
{\vcenter{\offinterlineskip\halign{\hfil$\scriptstyle##$\hfil\cr>\cr
\noalign{\vskip0.5pt}\sim\cr}}}
{\vcenter{\offinterlineskip\halign{\hfil$\scriptscriptstyle##$\hfil
\cr>\cr\noalign{\vskip0.5pt}\sim\cr}}}}}                               %
% Select actual type-setting size and redefine the makefootline-macro
%    (Defaults are:  \tenpoint    and   \baselineskip=24pt)
%
\twelvepoint
\def\makefootline{\baselineskip=36pt \line{\the\footline}}
%
% To get a title page without page number the macro  \nopagenumbers
% is used. Then \pageno is reset such, that the paper will start
% with page number 1 on the SUMMARY page (see the \summary macro below)
%
\nopagenumbers
%
% This initializes a set of counters for numbering purposes.
%
\newcount\chapnum
\chapnum = 0
\newcount\secnum
\secnum = 0
\newcount\subsecnum
\subsecnum = 0
\newcount\numeq
\numeq = 0
%
% Here we set and define some fundamental spacing parameters
%
\hsize    5.9in                      % default  6.5in
\vsize    8.9in                      % default  8.9in
\hoffset  0.4in
\voffset  0.0in
\parskip  5pt plus 1pt               % default  3pt plus 1pt minus .5pt
%
\def\chskipt{\vskip 20pt plus 0pt minus 0pt }
\def\chskipl{\vskip 5.5pt plus 0pt minus 0pt}
\def\secskipt{\vskip 6pt plus 2pt minus 1pt }
\def\secskipl{\vskip 3.5pt plus 1pt }
\def\subsecskip{\vskip 6pt plus 2pt minus 2pt }
\def\unchskip{\vskip -5.5pt }
%
%
% * * * * * * * * * * * * * * * * * * * * * * * * * * * * * * * * * *
%
%
% TITLE PAGE MACROS
%
% \pubdate{January 1983)     to get    January 1983
%                            placed in top right hand corner of
%                            title page
%
% \title{title of            The title will be centered and typed
%        paper\*}            in boldface type.  If the title is
%                            is longer than one line, use \title
%                            again.  \* will give you a superscript
%                            * following the title.
%
% \author{name}              The author's name will be centered
%
% \authort{name}             The names of two authors will be centered
%                            i.e  A.ONE and B.TWO
%                            with MPA's name and address below.
%
% \nextauthor{name}          To be used if you have more than
%                            one author - the name is centered.
%
% \institut{name}{address}{country}
% \institutb{name}{address}
% \institutc{name}{address}
%                            Institute name and address will be
%                            centered below the next author's
%                            name.  However, the } at the end of
%                            name and the { at the beginning of
%                            address must by on the same line with
%                            no space between them.  For example,
%                            "\institut{MPA}{Garching etc.}"
%
% \summary                   The word SUMMARY will be written left
%                            adjusted in boldface.
%
% \keywd{keywords}
%
% \submit{journal name}      to get   "Submitted to {journal name}"
%
% \aasec{section name}       according to A&A nomenclature
%
% \stitle{short title}       short title for A&A
%
% \thesaurus{numbers}        thesaurus number of A&A
%
% \offpri{name}
%
% \endpage                   same as \par \vfill \eject
%
%
% * * * * * * * * * * * * * * * * * * * * * * * * * * * * * * * * * *
%
\def\ctrline{\centerline}
%
\def\pubdate#1{\line{\hskip 12cm {#1}\hfill} \par \vfill}
%
\def\title#1{\ctrline {\bf #1} \medskip }
%
\def\author#1{  \ctrline{ \caps {#1} } }
%
\def\authort#1#2{  \ctrline{ {\caps {#1} \ and \ {#2}} }  \smallskip
    \ctrline{  \it {Max-Planck-Institut f\"ur Astrophysik} }
    \ctrline{ \it {Karl-Schwarzschild-Str. 1}                        }
    \ctrline{ \it {8046 Garching b. M\"unchen, FRG}                } }
%
\def\authora#1#2#3{ \ctrline{{\caps {#1} \ , \ {#2} \ , \ {#3} \ , \ }}}
%
\def\authorb#1{ \ctrline{\caps {#1} }
    \smallskip
    \ctrline{ $^2$ \it {Max-Planck-Institut f\"ur Astrophysik} }
    \ctrline{ \it {Karl-Schwarzschild-Str. 1}                        }
    \ctrline{ \it {D-8046 Garching b. M\"unchen, FRG}                } }
%
\def\nextauthor#1{ \smallskip
                   \ctrline{and} \par \ctrline {\caps {#1}} }
%
\def\institut#1#2#3{ \ctrline{\it #1} \ctrline{\it #2}
                     \ctrline{\it #3} \par \vfill }
\def\institutb#1#2{ \ctrline{$^1$ \it #1} \ctrline{\it #2} \par }
\def\institutc#1#2{ \ctrline{$^3$ \it #1} \ctrline{\it #2} \par }
%
% Because the macro \nopagenumbers was used above, we have to
% reset \footline and \pageno  on the second page to give it the
% page number 1. In addition the default \tenrm for the font of
% the page number is changed to \twelverm .
%
\def\summary{ \footline={\hss\twelverm\folio\hss} \pageno=1
              \line{\bf Abstract \hfill} \chskipl }
%
\def\submit#1{ \ctrline{Submitted to #1} }
%
\def\aasec#1{ \ctrline{Section: #1} }
%
\def\stitle#1{ \ctrline{Short title: #1} }
%
\def\thesaurus#1{ \ctrline{Thesaurus code numbers: #1} }
%
\def\offpri#1{ \par \vfill \ctrline{Send proofs and offprint
                                    requests to: #1} \vfill }
%
\def\keywd#1{ \line{ {\bf Key words:} #1\hfill } \vfill }
%
%
% * * * * * * * * * * * * * * * * * * * * * * * * * * * * * * * * *
%
%
% CHAPTER and SECTION MACRO
%
% \chap{chapter name}        This command numbers the chapter,
%                            generates part of the number to be
%                            used in defining equation numbers,
%                            left adjusts the line and prints it in
%                            boldface type.  The lack of space
%                            between \chap and { } is important.
%
% \titcon{text}              used to continue a chapter title to a
%                            second line if you cannot fit it all on
%                            one line.
%
% \sec{name}                 The \sec macro sets the name in upper
%                            and lower case capitals, prints the
%                            section name and number (left adjusted)
%                            and skips the appropriate amount of space.
%
% \subsec{name}              The \subsec macro allows you to title
%                            subsections with an underlined phrase
%                            which will be set on the same line as
%                            the first sentence of the first para-
%                            graph of the subsection.
%
%
% * * * * * * * * * * * * * * * * * * * * * * * * * * * * * * * * *
%
%\def\chpreset{ \global \secnum=0     \global \numeq=0 }
% -- this is for numeration of eqs. starting anew in every chapter
\def\chpreset{ \global \secnum=0 }
% -- this is for continuos numeration of eqs.
\def\secreset{ \global \subsecnum=0 }
%
\def\chap#1{ \global \advance\chapnum by 1  \chpreset \chskipt
             \line{\bf \the\chapnum . #1\hfill}
             \penalty 100000 \chskipl \penalty 100000 }
%
\def\titcon#1{ \unchskip \line{\bf{#1} \hfill}
               \penalty 100000 \chskipl \penalty 100000 }
%
\def\sec#1{ \global \advance\secnum by 1  \secreset \secskipt
    \line{\bf \the\chapnum .\the\secnum{ }{ }#1\hfill}
             \penalty 100000  }
%
\def\subsec#1{ \global \advance\subsecnum by 1  \secskipt
    \line{\bf \the\chapnum .\the\secnum .\the\subsecnum{ }{ }#1\hfill}
             \penalty 100000  }
%
%
% * * * * * * * * * * * * * * * * * * * * * * * * * * * * * * * * *
%
%
% EQUATION NUMBER MACRO
%
% \eqno (any number)         In display math mode, i.e. between
%                            $$...$$, you can use \eqno (any
%                            number), i.e.
%                            "$$...equation... \eqno (43)$$"
%                            gives you (43) on the right hand
%                            side of the page next to the equation.
%
% \eqno{\eqmm}               You can use \eqno{\eqmm} in display
%                            math mode to number the equations
%                            sequentially.  The number will appear
%                            on the right hand side of the page.
%                            TEX automatically numbers the equations
%                            for you. Use
%                            "$$ ...equation...\eqno{\eqmm} $$".
%                            {\eqmm} is a synonym for {\enum}.
%
% \labeq{\name}              In display math mode, if you wish to
%                            generate a dummy name for an equation,
%                            use \labeq{\name} as the argument.  From
%                            that point on \name will always be
%                            associated with that particular equation,
%                            and when \name is typed TEX will convert
%                            it to the appropriate number.
%
%                            The virtue of the dummy name is that if
%                            equations are added at a later time,
%                            named equations or equations numbered
%                            using \eqno{\eqmm} will be automatically
%                            correctly renumbered.  If \name is used
%                            then all textual references to that
%                            equation will also be renumbered.
%
%
% * * * * * * * * * * * * * * * * * * * * * * * * * * * * * * * * *
%
\newread\flabs
\ifcase\giveeqname
  \openout\flabs=flabs
\or
  \openin\flabs=flabs
\fi
%
\def\finlabs{
\ifcase\giveeqname
  \closeout\flabs
\or
  \closein\flabs
\fi
  }
%
%\def\eqname#1{ \global \advance\numeq by 1
%  \xdef#1{(\the\chapnum .\the\numeq )} (\the\chapnum .\the\numeq ) }
%\ifcase\giveeqname
  \def\eqname#1{ \global \advance\numeq by 1
   \xdef#1{(\the\numeq )} (\the\numeq ) }
%\or
%  \def\eqname#1{ \global \advance\numeq by 1
%    \xdef#1{(\the\numeq )}<{\bf#1}>(\the\numeq ) }
%\fi
%
%\def\enum{ \global \advance\numeq by 1 (\the\chapnum .\the\numeq ) }
\def\enum{ \global \advance\numeq by 1 (\the\numeq ) }
%
\def\eqmm{ \enum }
%
\def\refeq#1{ \xdef#1{(\the\numeq)}  }
%#1
%\#1        \grmm  \polsa             }
%
%\def\dumbs{ ^^ }
%
\ifcase\giveeqname
  \def\labeq#1{
  \write\flabs{\noexpand#1}
  \eqno \eqname{#1}
  }
\or
%  \def\labeq#1{ \eqno <{\bf#1}>\eqname{\#1} }
 \def\labeq#1{
   \global \advance\numeq by 1
   \refeq{#1}
   \global\read\flabs to\labl
   \eqno<{\bf \labl}>(\the\numeq)
                  } %  end of @def@labeq
\fi
%
%  {\catcode`\@=0
%   @catcode`@\=10
%   @def@labeq#1{
%    @global @advance@numeq by 1
%    @refeq{#1}
%    @global@read@flabs to@labl
%    <{@bf @labl}>
%    @eqno(@the@numeq)
%                   } %  end of @def@labeq
%    }  % end of catcod
%  {
%    \catcode`\@=0
%    @catcode`@\=10
%    @def@a{bbbbbbb}   @a   }
% {\rm
% \catcode`^^M=2
% \catcode`\/=0
%  \gdef\subl{\noexpand#1}
% /global/read/flabs to/labl
% \global\read\flabs to\labl
% /labl
% }
% \eqno<{\bf \labl}>(\the\numeq)
%
%
% * * * * * * * * * * * * * * * * * * * * * * * * * * * * * * * * * *
%
%
% BULLET, STAR AND DASH ITEM
%
% This is similar to the Point List Macro except that you
% get a bullet, a star or a dash instead of a number.
%
% \bullitem text             Type the text beginning on the same line
%                            (with a space between \bullitem and
%                            whatever follows).  You can also begin
%                            the text on the following line.
%                            The text will be indented, preceded by
%                            a bullet.  To end the text, type \endlist.
%                            You can type \par or leave a blank line
%                            and get the same result.
%
% \staritem text             The same as \bullitem above.
% \dashitem text             The same as \bullitem above.
%
% \sbullitem text            This command will give you a sub-bullitem.
% \sstaritem text            You will get a sub-staritem.
% \sdashitem text            You will get a sub-dashitem.
%
% \ssbullitem text           You get sub-sub-bullitem.
% \ssstaritem text           You get sub-sub-staritem.
% \ssdashitem text           You get sub-sub-dashitem.
%
% \endlist                   Ends the list.
%
%
% * * * * * * * * * * * * * * * * * * * * * * * * * * * * * * * * * *
%
\def\point{   \par
              \hangindent\parindent  \textindent  }
%
\def\spoint{  \par \indent
              \hangindent2\parindent \textindent  }
%
\def\sspoint{ \par \indent \indent
              \hangindent3\parindent \textindent  }
%
\def\bullitem{   \point{$\bullet $}}
\def\staritem{   \point{$\ast $}}
\def\dashitem{   \point{---}}
\def\sbullitem{  \spoint{$\bullet $}}
\def\sstaritem{  \spoint{$\ast $}}
\def\sdashitem{  \spoint{---}}
\def\ssbullitem{ \sspoint{ $\bullet $}}
\def\ssstaritem{ \sspoint{ $\ast $ }}
\def\ssdashitem{ \sspoint{ ---}}
\def\endlist{\par}
%
%
% * * * * * * * * * * * * * * * * * * * * * * * * * * * * * * * * *
%
%
% USEFUL COMMANDS
%
% \app{App}                  gives you APPENDIX A, centered and typed
%                            in boldface type.
%
% \ack                       ACKNOWLEDGMENTS will be centered and
%                            typed in boldface type.
%
% \endpage                   same as \par \vfill \eject
%
% \ket{\symbol}              gives you "| symbol >"
% \bra{\symbol}              gives you "< symbol |"
%
%
% * * * * * * * * * * * * * * * * * * * * * * * * * * * * * * * * *
%
\def\app#1{  \chskipt \line{\bf Appendix #1\hfill }\penalty 10000
             \chskipl \penalty 10000  }
%
\def\ack{  \chskipt \line{\bf Acknowledgements \hfill }\penalty 10000
           \chskipl \penalty 10000  }
%
\def\ket#1{\leftv #1\rangle}
\def\bra#1{\langle #1\rightv}
%
\def\endpage{\par \vfill \eject}
%
%
% * * * * * * * * * * * * * * * * * * * * * * * * * * * * * * * * *
%
%
% START MACROS
%
% If you wish to start printing a file in the middle and want the
% page numbers to come out correctly start your input file with
% \startpage{arabic number}
%
% \startpage{number}         The page count will start with the
%                            given number.
%
% \startchapter{number}      The paper will be printed starting with
%                            the given chapter number.
%
%
% * * * * * * * * * * * * * * * * * * * * * * * * * * * * * * * * *
%
\def\startpage#1{\count0 = #1}
\def\startchapter#1{\count\chapnum = #1}
% Um ein Symbol ueber ein anderes zu setzen:
%        \ueber{Grundsymbol}{Daruebergestelltes}
% Gedacht fuer die Verwendung in Formeln. Das Daruebergestellte wird
% in ganz kleiner Schrift gedruckt (\scriptscriptstyle).
%
\def\ueber#1#2{{\setbox0=\hbox{$#1$}%
  \setbox1=\hbox to\wd0{\hss$\scriptscriptstyle #2$\hss}%
  \offinterlineskip
  \vbox{\box1\box0}}{}}
\hsize 6.0in
\vsize 9.5in
\hoffset  0.10in
\voffset -0.0in
\vskip 1.0in
\baselineskip=18pt


\font\klein=cmr7
\font\mittel=cmr10
\font\normal=cmr10
\font\mathnormal=cmmi10
\font\mathklein=cmmi5
\font\mathmit=cmmi10
\font\kleinind=cmr5
\font\midind=cmmi10
\overfullrule=0pt
{\catcode`\@=11
\gdef\SchlangeUnter#1#2{\lower2pt\vbox{\baselineskip 0pt \lineskip0pt
  \ialign{$\m@th#1\hfil##\hfil$\crcr#2\crcr\sim\crcr}}}
  % kopiert von \@vereq aus dem TeXbook, Seite 360.
}
\def\gtrsim{\mathrel{\mathpalette\SchlangeUnter>}}
\def\lesssim{\mathrel{\mathpalette\SchlangeUnter<}}
\parskip=5pt
\def\ref#1{\lbrack #1\rbrack}
\def\eck#1{\left\lbrack #1 \right\rbrack}
\def\rund#1{\left( #1 \right)}
\def\ave#1{\langle #1 \rangle}
\def\wave#1{\left\lbrace #1 \right\rbrace}
\def\ueber#1#2{{\setbox0=\hbox{$#1$}%
  \setbox1=\hbox to\wd0{\hss$\scriptscriptstyle #2$\hss}%
  \offinterlineskip
\def\underline{\underline}
\def\x{x = x^{(m)}}
\def\ax{\underline{x} = \underline{x}^{(m-1)}}
\vbox{\box1\kern0.4mm\box0}}{}}
\def\ref#1{\lbrack #1\rbrack}
\def\eck#1{\left\lbrack #1 \right\rbrack}
\def\rund#1{\left( #1 \right)}
\def\ave#1{\langle #1 \rangle}
\def\wave#1{\left\lbrace #1 \right\rbrace}
\def\ueber#1#2{{\setbox0=\hbox{$#1$}%
  \setbox1=\hbox to\wd0{\hss$\scriptscriptstyle #2$\hss}%
  \offinterlineskip
  \vbox{\box1\kern0.4mm\box0}}{}}
\parindent=12pt
\overfullrule=0pt
%
 \def\etal{{et al.} \thinspace}
 \def\eg{{e.g.,} \thinspace}
 \def\ie{{i.e.,} \thinspace}
 \catcode`\@=11
%
 \def\ni{$^{56}{\rm Ni}\ $}      %Ni56 symbol
 \def\co{$^{56}{\rm Co}\ $}      %Co56 symbol
 \def\fe{$^{56}{\rm Fe}\ $}      %Fe56 symbol
%
 \def\kelvin{\thinspace\rm{\sp{o}{\kern-.08333em }K}\ }
                                 %degrees Kelvin symbol (math mode)
%-----------------------------------------------------------------------
%
% \def\ms#1{$#1$~M$_\odot$}
 \def\ms{\,\, {M_{\odot}}}
% \def\rs#1{$#1$~R$_\odot$}
 \def\rs{\,\, {R_{\odot}}}
 \def\mag#1#2{$#1^{m}\!\!\!.\,#2$}
%
%
%
\def\unvskip{%
   \ifvmode
      \ifdim\lastskip=0pt
      \else
         \vskip-\lastskip
      \fi
   \fi}
\newskip\tabefore \tabefore=20dd plus 10pt minus 5pt      % space above
\newskip\taafter  \taafter=10dd                           % space below
%
\def\begref#1{\par
   \unvskip
   \goodbreak\vskip\tabefore
   {\noindent\bf\ignorespaces#1%
   \par\vskip\taafter}\nobreak\let\INS=N}
\parindent=1.5em
\newdimen\stdparindent\stdparindent\parindent
\def\ref{\if N\INS\let\INS=Y\else\goodbreak\fi
   \hangindent\stdparindent\hangafter=1\noindent\ignorespaces}
\def\endref{\goodbreak}
%

